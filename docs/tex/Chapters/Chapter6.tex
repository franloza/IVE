\lhead{\emph{Personal experience}} 
\chapter {Personal experience}

I have to say that thanks to this project, I have learnt all the
difficulties that facing a real machine learning problem implies. We
were forced to move from the comfort of the guided practices of class to
the unpredictable nature of a real case study. In my case, these
difficulties almost put an end to my hopes of getting interesting
conclusions from my final project, which has been a priority for me from
the very beginning.

My first final project proposal was DJIA (Dow Jones Index Analyzer), an
attempt for controlling in some manner the unpredictable world of the
stock exchange. From the beginning, I thought that I would be a
difficult task, not only because of the difficulty of the problem
itself, but because the paper that referred to this dataset was about
simple statistics and not machine learning. Taking into consideration
this difficulties, I tried to build a framework as modular as possible
in order to have an ``emergency exit'' if the project resulted to be a
disaster.

Indeed, it was a complete disaster. I couldn't get more than 51\% of
accuracy and the precision-recall graphics brought me back down to
earth. Predicting the stock was impossible.

That led me to the situation of having to start from scratch with a new
dataset. However, this situation was mitigated thanks to the modularity
of my project. In less than 48 hours, I had my new dataset working and
showing up more than interesting results. This was the real origin of
BMML.

I wanted to come up with my own conclusions even if I was wrong and
don't conditioning myself too much for the results of the study paper. I
thought that I would be more interesting to apply the techniques we saw
in class and getting my own conclusions. That's the reason because my
paper is not very similar to the Moro's, Cortez's and Rita's one.
Nevertheless, I think I have got interesting conclusions and I tried to
give an explanation for all of them in a clear manner.

Finally, I tried to construct a framework of machine learning gathering
all the practices we did in the laboratory, adding parameters to
customize the process (Up to 15 different parameters and 3 available
algorithms). This project was open since the very beginning and I
encouraged to my classmates to use it in their projects so as they could
get interesting conclusions and make interesting projects. This
repository is available both in the
\href{https://github.com/franloza/BMML}{Github repository} and the
\href{http://franloza.github.io/BMML/}{webpage} of the project.

Also, I have to thank to my classmate \'Alvaro Bermejo, because without him
and our effort for finishing all the practices, it's very likely that
this project wouldn't have gone so far. I want to continue learning more
techniques of machine learning and try other tools that we have not used