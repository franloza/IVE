\lhead{\emph{Conclusions}} 
\chapter{Conclusions}

This experiment led us to face several problems and to learn about the right
approach to answer a research question like this. The first conclusion that we
draw is that the actual results are rather different from the ones
we expected. As explained above, we can explain these differences by several flaws in the experiment.
Analyzing the results statistically, we assert that the distortion from
our expections could be due to the small size of our dataset. This issue
could not be foreseen at the beginning, but it turned out to be a
problem at the end.

The second conclusion is that Virtual Reality offers as many solutions
as challenges. Dealing with the deadline for this experiment was
very challenging because of the technical issues related to the
compatibility of the Oculus SDK and the Unity framework. This forced
us to perform the experiment on a small number of people and limited our
capacity to extract significant conclusions from the dataset.

Moreover, the limited time affected the behavior of the subjects. Some
of them misenderstood the task to accomplish, influencing negatively the
results or forcing us to discard the results. On the other hand the
subject that was able to accomplish the tasks increased his performances
with the growth of the difficulty. This was an unpredictable training
process that we didn't take into account. The second unexpected process
was the tendency of the subjects to spend more time to enjoy the VR
experience, and not because of it helped in the comprehension of the
graph, producing some distortions on the data.

Despite the difficulties, we have derived two conclusions from the data:
VR and 3D vision with headtracking positively affects the visualizations
of 3D graphs and subjects prefer immersive environment and within it
they obtain better performances. Thanks to the problems we faced, we
spreaded our knowledge not only about software development (Unity
developing, Oculus Rift Hardware, Computer Graphics basics etc.) but
especially about science (scientifically approach, experimental issues,
recuiting subjects, evaluating other experiments, being subjects of other
groups etc.).

Finally we thought about some possible applications of this results in
case they are supported by a bigger number of datapoints and a more solid
scientific approach. This kind of experiment could be implemented in the
diagnosis and the treatment of neurogical deseases and in the rehab
processes measuring the level of the interactions of the patients and
their improvements during the period. In this scenario it would also be
useful to run the third stage of our experiments that would try to analyze the
effect on the visual memory of a restricted field of view.





