\lhead{\emph{Introduction}} 
\chapter{Introduction}

Much research about Virtual Reality has proved that the combination of a
perspective stereoscopic view and tracking the position of
the user in real time as he moves is a good idea for visualizing 3D scenes. One of the
papers that has gone deeper in the topic of visualizing 3D objects using
VR is \emph{Evaluating Stereo and Motion Cues for Visualizing
Information Nets in Three Dimensions \cite{ware1996evaluating}}.

The main goal for this project is to replicate one of the experiments
carried out in the abovementioned paper. This experiment consists of
analyzing the difference between user comprehension of network graphs
(1) in 2D versus 3D and (2) with and without head-tracking. This
difference will be measured by having the user perform a simple task and
recording several metrics such as timing and error rate. As a result of
the experiment, we hope to be able to make conclusions about whether 3D
vision and/or (lack of) head-tracking is better for the comprehension of
visualized graphs.

In addition to the metrics proposed in the experiment, we will try to
gain more insight into the effects of 3D vision and head-tracking by
measuring additional metrics such us the distance traversed or the
amount rotated by the head or the distance that the subject traverses
inside the virtual environment. The purpose of these additional metrics
is to test whether or not there exists a correlation between the
complexity of the graph and the amount of movement that is required for
the subject to comprehend it. Of course, our purpose is not to provide a
completely general answer about how much is gained by using
head-tracking because its use is task-specific. The conclusions will
only try to generalize a hypothetical application in real life for
visualizing graphs and they should not be considered as applicable in
all scenarios where head-tracking is used.

As students, we are conscious of many problems that may arise during the
development of the experiment and we will have to overcome some
difficulties that could lead to the failure of the experiment. Some of
these difficulties are:

\begin{itemize}
\item
  Retrieve some metrics from Oculus Rift and get used to the Unity
  framework.
\item
  Design appropriate levels (of difficulty) for our analysis purpose in
  order to get useful data.
\item
  Interpret the data correctly to extract truthful and correct
  conclusions.
\item
  Technical difficulties/limitations regarding frame rate, graphics
  rendering, etc.
\end{itemize}

All these risks are taken into account and the priority of this study is
not only to understand some of the topics we have seen in class but also to
see the difficulties that may arise when dealing with a Virtual Reality
experiment.

To define a guiding line for our experiment, we have defined a series of
questions that we want to answer:

\begin{itemize}
\item
  Is it better in terms of comprehension to visualize network graphs in
  2D or 3D?
\item
  Does head-tracking help increase this comprehension in addition to 3D
  vision?
\item
  Do these differences become bigger as we increase the complexity of
  the graph?
\item
  How similar are the obtained results to those in the original paper?
\end{itemize}

Therefore, we will try to provide quantitative measures that could prove
how much more a graph can be understood in 3D compared to 2D. At the
end of the report, we give strong reasons about whether using 3D
graphics and virtual reality is a good combination to interact with
information structures.
