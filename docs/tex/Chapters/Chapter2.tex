\lhead{\emph{Related Work}} 
\chapter{Related Work}

As we have mentioned in the previous section, this experiment is an
extension of the one proposed in the study \emph{Evaluating Stereo and
Motion Cues for Visualizing Information Nets in Three Dimensions.}
Despite the paper of Ware and Franck is a good reference for comparing
our conclusions with theirs, we must consider the context and the
technology used for doing their experiment. This paper was published in
1996, when virtual reality started to be used in some applications but
the technology was not very advanced in this area and a lot of research
tried to show the possible future applications it may have in the next
decade. The hardware used in the paper consisted of 3D LCD shutter
glasses that provided the stereo vision and an ultrasonic head-tracking
device embedded inside the frame of the glasses. The monitors used for
the experiment ran at 120Hz and each eye received a 60Hz update rate
using the shutter glasses.

Of course, a great advanced in technology has happened since 1996, and
twenty years later, more sophisticated technology is used, and this is
an important factor that we cannot overlook. For our experiment, we use
Oculus Rift Development Kit 2. This device has a resolution of 960x1080
per eye, a low-persistence OLED display, positional tracking and higher
refresh rate that previous versions of Oculus Rift (Up to 75 Hz) and an
external HDMI port.

The difference in technology is one of the main differences between our
experiment and Ware and Franck's experiment. We introduced some
differences in the conditions and the variables of the experiment. In
the original paper, four different combinations in the type of
visualization are carried out: 2D, 2D and head tracking, 3D and 3D and
head tracking. However, in our experiment, only three conditions are
tried: 2D, 3D and head tracking, and a combination of 3D, head tracking
and partial visual reduction. This last condition could not be performed
due to technical issues with the development framework and the time
constraints of the experiment.

Moreover, the variables taken into account for performing the experiment
also changed. In the original paper, the number of nodes are varied to
measure the error rate and the time employed by the subjects. They
extract conclusions about performance using this metrics. On the other
hand, we used a different approach for obtaining different results
instead of the raw number of nodes and an abstraction called
\textbf{complexity level} is used instead. This abstraction is made
taken into account the number of nodes and the minimum and maximum
degree that they may can have. This degree may vary among the define
range but the number of nodes is fixed for each complexity level. Using
different types of complexity levels, we tried to simulate different
scenarios of complexity and how much does it affect to the error rate
and employed time to comprehend the graph.

Finally, the head movement is also taken into account. We use a distance
metric provided by the Unity framework in a virtual space that allows us
to compare among different amounts of movement. This variable is new
respect of the original paper and it's interesting to analyze because in
may provide information about how head-tracking affects the behavior of
the subjects and extract conclusions about it.

