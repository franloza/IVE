\lhead{\emph{Experiment}} 
\chapter{Experiment}

\section{General description}\label{general-description}

The experiment has been design with the main purpose to collect relevant
and consistent data from the Oculus Rift device and the subjects
regarding the effects of stereoscopic vision and head tracking in the
task of 3D graph comprehension. In addition, we will create a series of
scenarios to retrieve data regarding to the effects of visual memory in
3D graph comprehension when the range of vision is reduced.

The experiment will be composed of \textbf{three stages}. Each one will
be composed of \textbf{three scenarios} according to the graph
complexity\textbf{.} The description of each stage is the following one:

\begin{enumerate}
\def\labelenumi{\arabic{enumi}.}
\item
  \textbf{Stage 1}: The subject performs the experiment in each scenario
  directly on the screen of a laptop. For this stage, no Oculus Rift
  device will be used and the subject will be able to visualize the
  graph looking at the screen, with no head tracking support.
\item
  \textbf{Stage 2}: The subject performs the experiment in each scenario
  using the Oculus Rift device. Head tracking will be active and there
  will not be any kind of reduction of the angle of vision.
\item
  \textbf{Stage 3}: The subject performs the experiment in each scenario
  using the Oculus Rift device, Head tracking will be active and there
  will be a reduction of \textasciitilde{}50\% of the range of vision
  both in horizontal and vertical axes.
\end{enumerate}

Each stage will be composed by three scenarios with \textbf{challenges}
whose level of difficulty goes from 1 to 3. A challenge will consist of
answering if there exists a 2-length path connecting two highlighted
nodes. Each challenge will have a 50\% of probability of containing such
a path. The subject will have to answer (yes or not) whether or not this
path exists before the maximum time for the challenge is over. If the
user has not made a decision before the time is over, it will count as a
fail.

The complexity of the graph will be defined in terms of the number of
nodes and edges in the graph. Each scenario will contain a graph with
different complexity in increasing order. Due to the difficulty to
determine the number of nodes and edges that fit best with each level of
complexity, some testing will still be performed using different number
of nodes and edges to design three complexity levels that have distinct
average times needed to solve the problem in the experiment.

The estimated mean time to complete the three stages will depend on the
maximum time we define for each level of complexity but it will not be
more than 10 minutes. The estimated time for each stage will be in a
range between 1 and 3 minutes and a mean time of 1 minute between each
stage.

Finally, the experiment will be allowed to be repeated several times by
a subject. The first attempt will be considered for the main study and
the consecutive attempts will be taken into account for a possible
secondary study about the adaptive process in each stage.

\section{Constraints}\label{constraints}

For this experiment, we have defined some rules and constraints that
will delimit the experiment's boundary and specify clearly what the
subject can or can't do. The constraints are the following:

\begin{itemize}
\item
  A maximum amount of time will be defined for each level of complexity.
  The range of maximum time will be somewhere between 10 seconds and 1
  minute. This is an approximation and could be modified after the
  implementation of the experiment.
\item
  If the subject runs out of time for one level, the next level or stage
  will start and the challenge will be treated as failed.
\item
  The subject won't be allowed to stand up from the chair. He will be
  able to rotate his head and move it at any direction though.
\item
  A subject will be allowed to repeat the experiment a maximum of 3
  times to avoid virtual reality sickness.
\item
  The subject will sit in a range between 60 centimeters and 1 meter in
  front of the laptop screen for Stage 1.
\item
  If the subject interrupts the experiment without completing all the
  stages and levels, the results will be discarded.
\item
  The subject won't be allowed to take out the Oculus Rift in Stages 2
  and 3. If the subject takes out the OR during this stages, the
  experiment will be interrupted and the data collected will be
  discarded.
\item
  The subject won't be provided with any joystick to move inside the
  scenarios. He will have access to the mouse and/or keyboard to provide
  a yes or no answer to each experiment.
\end{itemize}

\section{Research questions}\label{research-questions}

In this section we are defining the different approaches that we will
use to answer the research questions with the data collected in the
experiment. The proposed research questions are the following ones:

\begin{itemize}
\item
  \emph{Does stereoscopy and head tracking add to a person's
  comprehension of 3D graphs?}
\end{itemize}

This question will be answered using the data collected from Stages 1
and 2. We will compare the difference of performance in terms of time
and error rate. In addition, the subject will be asked at the end of the
experiment to rate the experience in each stage. Data collected in these
two stages will be compared among them to see if these techniques are
really helpful and how much are they in terms of performance.

\begin{itemize}
\item
  \emph{Does there exist any correlation between the amount of head
  movement and the graph complexity when 3D graphs are visualized with a
  virtual reality device?}
\end{itemize}

This question will be answered using the data collected from Stages 2
and 3. We will measure the distance traversed by the head for each
scenario and use this metric to see if the distance increases
proportionally to the complexity of the graph.

\begin{itemize}
\item
  \emph{How important is the range of vision in comprehension of 3D
  graphs with stereoscopic vision? Does visual memory have any influence
  in the comprehension of 3D graphs?}
\end{itemize}

The second question is the most relative one because it depends on many
factors that are hard to measure without special equipment. However, our
approach of reducing the range of vision of the graph could give us some
idea of how much the difference in the performance is in this scenario.
We hypothesize that the reduction of the range of vision in Stage 3 will
make the subject look several times in different directions to construct
a memory image of the graph. This probably takes a certain amount of
time that we will compare with the results obtained in Stage 2.
Analysing the results, we will try to form a conclusion on how
significant the use of visual memory is in the comprehension of 3D
graphs.

\begin{itemize}
\item
  \emph{How similar are the obtained results with the findings in the
  original paper?}
\end{itemize}

This question will be answered using the data collected from Stages 1
and 2 and once the first question has been answered. We will make a
comparison of the conclusions of our experiment and the paper and try to
look for the reasons if the results are differences are significant
enough.


