\chapter{Implementation}

For the implementation of this experiment, we have used Unity framework
and Oculus Rift SDK. A controller class was in charge of controlling the
experiment flow across the different stages and scenarios. The
controller contained an internal timer that controlled the remaining
time for each scenario. The set of rules, such as maximum time and
reduction of range of vision, used for each stage are stored in a data
structure.

The only interaction allowed with each scenario was provided through the
mouse. The subject will use the left mouse button to answer that the
there exists a 2-length path between the highlighted nodes and the right
mouse to give a negative answer. The rest of the interactions such as
initiating the timer, change the scenarios or reduce the angle of vision
is done through the controller and no movement is implemented inside the
scenario.

The graphs were randomly generated for each scenario using some defined
constants according to the level of complexity of the graph. Some of
this constants were the number of edges and nodes and the probability of
generating a 2-length path between two points. The graph was undirected,
with white edges connecting blue nodes. The highlighted points were
drawn in red and the background of the scenario will be black to improve
the visibility of the graph. Finally, all the results collected were
exported to a CSV file and analyzed using Python and matplotlib library.